\documentclass[11pt, onecolumn]{article}
\usepackage[spanish, activeacute]{babel}
\usepackage[utf8]{inputenc}


% Letra
\usepackage{amsfonts}
\topmargin 10pt
\advance \topmargin by -\headheight
\advance \topmargin by -\headsep
\textheight 8.5in
\oddsidemargin -0pt
\evensidemargin \oddsidemargin
\marginparwidth 0.5in
\textwidth 6.5in
\let \large=\normalsize
\let \Large=\large
\let \tilde=\widetilde

% Hipervínculos
\usepackage[
colorlinks = True,
linkcolor = blue,
filecolor = blue
urlcolor = blue,
citecolor = blue
]{hyperref}

% Matemática
\usepackage{amsmath}
\usepackage{amssymb}
\newtheorem{theorem}{Theorem}
\newtheorem{Assumption}{Assumption}
\newtheorem{lemma}{Lemma}
\newtheorem{problem}{Property}
\newtheorem{proposition}{Proposition}
\newenvironment{proof}[1][Proof]{\textbf{#1.} }{\rule{0.5em}{0.5em}}

% Tablas y figuras
\usepackage{multicol}%
\setcounter{MaxMatrixCols}{30}
\usepackage{graphicx}
\DeclareGraphicsExtensions{.pdf,.eps,.ps,.png,.jpg,.jpeg}
\renewcommand{\arraystretch}{.5}
\newtheorem{corollary}{Corollary}
\renewcommand{\arraystretch}{1.5}
\usepackage{lscape}
\providecommand{\U}[1]{\protect\rule{.1in}{.1in}}
\graphicspath{{uno_graphics/}{uno_tcache/}{uno_gcache/}}
\providecommand{\U}[1]{\protect \rule{.1in}{.1in}}

% Citado
\usepackage{harvard}

\makeatletter
\makeatother

% Datos
\title{Análisis de sentimientos en comentarios de Among Us}
\author{Mauricio Alvarado}
\date{\today}



% Recursos:
% Como datos de entrenamiento se usará el dataset traducido del corpus Stanford Sentiment
% Treebank (SST-2). El dataset se puede acceder desde  el siguiente url:
% https://huggingface.co/datasets/mrm8488/sst2-es-mt


\begin{document}



\section{Motivación}
% Explicar el app y el motivo para escogerlo, rango de fechas de los comentarios

\section{Hechos estilizados}
% Indicar estadísticas (total de comentarios, cantidad de símbolos y palabras)
% Frecuencia de palabras condicional (por score)
% Identificar cantidad de verbos y adjetivos en los comentarios
% ¿Qué personas u organizaciones son mencionados en los comentarios?

\section{Metodología}
% o	Entrenar al menos 2 algoritmos (1 tradicional y 1 profundo) que pueda identificar
% el sentimiento del comentario (positivo, negativo). Aplicar pre-procesamiento y
% vectorización de texto.

\section{Selección de modelo}
% Explicar el proceso del entrenamiento y de las pruebas (porqué escogieron esos
% algoritmos, qué métricas usaron y sus valores, cual obtuvo el mejor performance,
% problemas que se presentaron, etc.)

\section{Resultados}
% Brindar conclusiones
\section{Conclusiones}




\end{document}