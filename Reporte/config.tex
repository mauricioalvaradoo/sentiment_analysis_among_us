\documentclass[12pt, a4paper]{article}
\usepackage[spanish,activeacute]{babel}
\usepackage[utf8]{inputenc}
\usepackage[T1]{fontenc}
\usepackage{helvet} % Importando helvética
\renewcommand*\familydefault{\sfdefault} % Arial

\usepackage{apacite}
\usepackage{natbib} % Natbib

\usepackage{geometry}
\newgeometry{bottom=2.5cm, top=4cm, left=2.5cm, right=2.5cm}

\usepackage{ragged2e}
\renewcommand{\baselinestretch}{1.25}
\setlength{\parindent}{0pt}
\setlength{\parskip}{6mm}

\setlength{\skip\footins}{1.5cm}
\setlength{\footnotesep}{0.5cm}

\usepackage{graphicx}
\usepackage[font=small,labelfont=bf]{caption}
\usepackage[skip=0pt]{subcaption} \captionsetup[subfigure]{labelformat=empty}
\usepackage{float}
\usepackage{booktabs}
\usepackage[dvipsnames]{xcolor}

\usepackage{amsmath}

\usepackage[colorlinks = True, linkcolor = blue, urlcolor = blue, citecolor = blue]{hyperref}
\usepackage{xurl}


%%%%%%%%%%%%%%%%%%%%%%%%%%%%%%%%%%%%%%%%%%%%%%%%%%%%%%%%%%%%%%%%%%%%%%%%%%%%%%%%%%%%%%%%%%%%

\title{Análisis de sentimientos en comentarios de Among Us}
\author{Mauricio Alvarado}
\subtitle{Reporte}
\date{\today}


%%%%%%%%%%%%%%%%%%%%%%%%%%%%%%%%%%%%%%%%%%%%%%%%%%%%%%%%%%%%%%%%%%%%%%%%%%%%%%%%%%%%%%%%%%%%
% Referencias/Bibliografía

%\usepackage[babel]{csquotes}
%\usepackage[backend=biber,style=apa]{biblatex}
%\renewcommand*{\bibfont}{\normalfont\scriptsize} % Cambiar tamaño del texto
%\setlength\bibitemsep{1.5\itemsep} % Separación entre cada referencia
%\DeclareLanguageMapping{spanish}{spanish-apa}
%\addbibresource{<<ruta>>} % Importando biblio
%\nocite{*} % Incluir las que no se citó

% Dentro del main se debe colocar: \printbibliography
% https://tex.stackexchange.com/questions/131518/help-with-citation-in-text-in-parentheses-etc-with-biblatex-apa

%%%%%%%%%%%%%%%%%%%%%%%%%%%%%%%%%%%%%%%%%%%%%%%%%%%%%%%%%%%%%%%%%%%%%%%%%%%%%%%%%%%%%%%%%%%%
% Paquetes adicionales:___________________________________________________
% Para tablas complejas se puede usar la página: https://www.tablesgenerator.com/
% Para cortar tablas, el paquete: \usepackage{multicol}
% Para tablas que abarcan más de dos páginas: \usepackage{longtable}
% Para páginas rotar páginas, sin rotar en enunciado: \usepackage{pdflscape}, usar comando: {landscape}
% Colores de fondos, textos, entre otros: \usepackage{color}
% Para cargar imágenes de pdfs: \usepackage{pdfpages}

